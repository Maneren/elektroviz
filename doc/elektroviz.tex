% LTeX: language=cs-CZ
\documentclass[12pt,a4paper]{paper}
\usepackage{a4wide}
\usepackage[margin=2.5cm]{geometry}
\usepackage[czech]{babel}
\usepackage{hyphsubst}
\usepackage{graphicx}
\usepackage{indentfirst}
\usepackage{microtype}
\usepackage{float}
\usepackage{amsmath}
\usepackage{esvect}

\selectlanguage{czech}
\PassOptionsToPackage{hyphens}{url}

\usepackage{hyperref}
\hypersetup{
    colorlinks=true,
    linkcolor=black,
    filecolor=magenta,      
    urlcolor=blue,
    pdftitle={Elektroviz},
}
\urlstyle{same}

% define helper command for typesetting code
\newcommand{\code}[1]{\texttt{#1}}
\def\CC{{C\nolinebreak[4]\hspace{-.05em}\raisebox{.4ex}{\tiny\bf ++}}}

\chardef\_=`_

% increase spacing between table rows
\renewcommand{\arraystretch}{1.25}

\begin{document}

\begin{figure}[H]
	\centering
	\includegraphics[width=0.8\textwidth]{pic/kiv-cmyk-cz}
\end{figure}

\begin{center}
	\vspace{.5cm}
	\LARGE{Elektroviz - Vizualizace elektrického pole}\\
	\large{Semestrální práce - KIV/UPG 2024/25}
\end{center}

\vfill

\noindent
Západočeská univerzita v Plzni \hfill Pavel Altmann - A23B0264P\\
Katedra informatiky a výpočetní techniky \hfill 1. odevzdání\\
Stráveno hodin: 95 \hfill \today
\thispagestyle{empty}

\newpage
\setcounter{page}{1}

\tableofcontents

\newpage

\section{Zadání}

Úkolem semestrální práce je vytvoření interaktivního programu pro vizualizaci
elektrostatického pole vytvořeného bodovými elektrickými náboji rozmístěných v
prostoru.

Program bude možné spustit z příkazové řádky skriptem \code{./run.sh} s
volitelným celočíselným parametrem určujícím scénář (viz níže) k vizualizaci.
Nebude-li parametr zadán, bude zvolen scénář 0. Po spuštění programu se zobrazí
okno o minimální počáteční velikosti 800x600px, ve kterém se zobrazí přehledně
všechny bodové náboje zvoleného scénáře. Vizuální reprezentace náboje (v
nejjednodušší podobě kružnice s popiskem) musí být volena tak, aby bylo možné
jednoznačně určit velikost náboje a jeho znaménko. V místě sondy bude zobrazen
šipkou vektor intenzity vektorového pole 𝐸⃗ a opatřen popiskem udávající jeho
velikost $\|E\|$. Všechny elementy vizualizace (náboje, popisky, šipky) musí být
plně viditelné.

Intenzita vektorového pole v daném místě $\vec{x}$ se spočte
z Coulombova zákona jako:
\[ \vec{E}(\vec{x}) = \frac{1}{4 \pi \epsilon_0}
\sum_i q_i \frac{\vv{x_i - x}}{\|\vv{x_i - x}\|^3} \]

Vizualizace se bude pravidelně (s časem) aktualizovat. Jednotlivé elementy
vizualizace budou automaticky svou velikost a rozmístění přizpůsobovat velikosti
(a tvaru) okna tak, aby se prostor okna „maximálně“ využil, ale zároveň aby
nedošlo ke zkreslení souřadnic.

Scénáře:

\begin{itemize}
  \item Scénář 0 = pouze jeden bodový náboj +1C v počátku souřadnic (0, 0);
  \item Scénář 1 = jeden náboj +1C v (-1, 0), druhý +1C v (1, 0);
  \item Scénář 2 = jeden náboj -1C v (-1, 0), druhý +2C v (1, 0);
  \item Scénář 3 = čtyři náboje: +1C v (-1, -1), +2C v (1, -1), -3C v (1, 1)
    a -4C v (-1, 1);
\end{itemize}

Ve všech těchto scénářích se sonda pohybuje po kružnici se středem v počátku
souřadnic (0, 0), poloměrem 1 a úhlovou rychlostí \( \frac{\pi}{6}
\frac{rad}{s} \).

\section{Implementace řešení}

\subsection{Použité technologie}

Vizualizace je naprogrogramována v jazyce \CC{} 23 s využitím 2D grafické
knihovny \href{https://www.raylib.com/}{Raylib} verze 5.5.


\subsection{Architektura}

Program využívá OOP architekturu, kde každý grafický a logický prvek je uzavřený
do své třídy a zodpovědný za veškeré výpočty a kreslení, které se ho týkají.
Funkce \code{main} má na starost pouze počáteční nastavení a následné předávání
událostí (nový snímek, změna velikosti okna, apod.) objektům. Některé objekty
jako součást obsahují další objekty a potom mají na starost jim události
předávat.

\subsection{Objekty}

Každý objekt má především 2 metody \code{update} a \code{draw}. \code{update} je
zodpovědná za přepočítání vlastností objektu v závislosti na čase, popř. změně
vlastností jiných objektů (např. sonda mění polohu podle času a naměřenou
hodnotu podle síly a polohy nábojů).

Každý zde popsaný objekt se nachází ve svém vlastním souboru.

\subsubsection{\code{Charge}}

Reprezentuje v simulaci jeden náboj. 

Velikost náboje může být buď konstantní nebo proměnlivá v čase - reprezentováno
třídami \code{StaticStrength} a \code{VariableStrength}. 

Vykresluje se jako kruh, jehož plocha (nikoliv průměr) je přímo úměrná velikosti
náboje. Dále má bílé ohraničení a barvu výplně reprezentující velikost náboje -
červená 5 C a více, modrou 5 C a méně a lineární interpolaci přes tmavě šedou
pro hodnoty mezi tím.

\subsubsection{\code{Probe}}

Reprezentuje v simulaci elektrickou sondu.

Může být buď statická nebo otáčivě pohyblivá - reprezentováno třídami
\code{Static}, resp. \code{Rotating} ze jmenného prostředí \code{position}.

Vykresluje se jako šipka reprezentující směr. Délka šipky je pro přehlednost
pevně daná, protože pokusy s proměnnou délkou v závislosti na intenzitě pole se
neosvědčili. Intenzita se pohybuje v rozmezí několika řádů a velikost se tedy
pohybovala od sotva viditelné po více než 3 obrazovky dlouhé.

Zároveň může sonda volitelně zobrazovat text s numerickou velikostí intenzity ve
vědecké notaci.

\subsubsection{\code{Grid}}

Představuje mřížku na pozadí, včetně sond v jejích průsečících.

Má nezávisle nastavitelné rozestupy mezi čárami v obou směrech v pixelech - ve
výchozím nastavení to je 50x50. Je dobré nenastavovat příliš malé rozestupy,
protože pak je jednak obraz nepřehledný a druhak se velmi zvyšují nároky na
výpočetní výkon grafické karty.

Pro zrychlení kreslení se potřebné souřadnice čar i sond počítají pouze při
inicializaci a změně rozměrů okna a jinak jsou uložené v soukromých atributech
třídy.

\subsection{Ostatní soubory}

\subsubsection{\code{field}}

Jmenné prostředí obsahuje funkce \code{E} a \code{potential}, počítající vektor
intezity, resp. skalární potenciál elektrického pole v zadaném bodě.

\subsubsection{\code{utils}}

Obsahuje pomocné funkce pro interpolaci barev \code{lerpColor} a \code{lerpColor3},
funkci \code{sigmoid} pro počítání stejnojmenné matematické funkce a funkci
\code{trim} pro odstranění počátečních a koncových bílých znaků z řetězce.

\subsubsection{\code{BoundingRectangle}}

Pomocná třída pro škálování.

Umožnuje snadné spočítání nejmenšího obdélníku, který obsahuje všechny grafické
objekty. Na základě jeho rozměrů se obraz škáluje, aby strany tohoto obdélníku 
byly zhruba polovina okna.

\subsubsection{\code{MathEval}}

Sada funkcí pro vyhodnocovaní libovolných matematických výrazů, využívaná pro
dynamicky nastavitelné vlastnosti. Další info patří do pozdějšího odevzdaní.

\subsection{Algoritmy}


\section{Ovládání programu}

\subsection{Sestavení}

Program je kompatibilní s GNU/Linux systémy s podporou grafického protokolu X11.

Sestavení se provádí pomocí přiloženého \code{Makefile} příkazem \code{make
setup build}.

Protože ale mnoho systémů nemá nainstalované hlavičkové soubory pro knihovnu X11
(potřebné pro vývoj většiny grafických aplikací), má zastaralé verze překladačů,
apod. zvolil jsem použití systému Docker. Sestavovací skript \code{./build.sh}
vytvoří kontejner z přiloženého souboru \code{Dockerfile} a spustí ho. Ten 
spustí přiložený \code{Makefile}, který do složky \code{vendor} stáhne knihovny
a vše sestaví do složky \code{build}. Skript pak ještě přesune výsledný
spustitelný soubor do souboru \code{bin/elektroviz.new}.

\subsection{Spuštění}

Program přebírá z příkazové řádky parametr \code{scenario} a volitelně
\code{spacing}. \code{scenario} Určuje který scénář má být spuštěn po zapnutí
aplikace, při absenci parametru se použije scénář 0. \code{spacing} určuje
rozestupy mřízky na pozadí ve formátu \code{<width>x<height>}.

\textbf{Důležité}: Program je nutné spustit ve stejné složce, v jaké se nechází
složka \code{scenarios}, jinak nebude schopný načíst žadný z výchozích scénářů.
Originálně je v \code{src} ale \code{./build.sh} jí pro snadnější manipulaci
nakopíruje i do kořenové složky.

\section{Limitace}

Program pouze minimálně ošetřuje nesprávné uživatelské vstupy a reaguje na ně
vyjímkami, které většinou způsobují pád aplikace. 

\end{document}
