% LTeX: language=cs-CZ
\documentclass[12pt,a4paper]{article}
\usepackage{a4wide}
\usepackage[margin=2.5cm]{geometry}
\usepackage[czech]{babel}
\usepackage{hyphsubst}
\usepackage{graphicx}
\usepackage{indentfirst}
% \usepackage{microtype}
\usepackage{float}
\usepackage{amsmath}
\usepackage{esvect}
\usepackage{tikz}
\usetikzlibrary{shadows}

\selectlanguage{czech}
\PassOptionsToPackage{hyphens}{url}

\usepackage{hyperref}
\hypersetup{
    colorlinks=true,
    linkcolor=black,
    filecolor=magenta,      
    urlcolor=blue,
    pdftitle={Elektroviz},
}
\urlstyle{same}

% define helper command for typesetting code
\newcommand{\code}[1]{\texttt{#1}}
\def\CC{{C\nolinebreak[4]\hspace{-.05em}\raisebox{.4ex}{\tiny\bf ++}}}

% helper for typesetting keystrokes
\newcommand*\keystroke[1]{
  % \hspace{0.2ex}
  \tikz[baseline=(key.base)]
    \node[draw,
          text height=1.5ex,
          text depth=0ex,
          fill=black!10,
          drop shadow={shadow xshift=0.2ex,shadow yshift=-0.2ex,fill=black,opacity=0.50},
          rectangle,
          rounded corners=2pt,
          inner sep=2.75pt,
          line width=0.5pt,
          font=\footnotesize\sffamily
    ](key) {#1};
  % \hspace{0.4ex}
}

\chardef\_=`_

% increase spacing between table rows
\renewcommand{\arraystretch}{1.25}

\begin{document}

\begin{figure}[H]
	\centering
	\includegraphics[width=0.8\textwidth]{pic/kiv-cmyk-cz}
\end{figure}

\begin{center}
	\vspace{.5cm}
	\LARGE{Elektroviz - Vizualizace elektrického pole}\\
	\large{Semestrální práce - KIV/UPG 2024/25}
\end{center}

\vfill

\noindent
Západočeská univerzita v Plzni \hfill Pavel Altmann - A23B0264P\\
Katedra informatiky a výpočetní techniky \hfill 2. odevzdání\\
Stráveno hodin: 120 \hfill \today
\thispagestyle{empty}

\newpage
\setcounter{page}{1}

\tableofcontents

\newpage

\section{Zadání}

Úkolem semestrální práce je vytvoření interaktivního programu pro vizualizaci
elektrostatického pole vytvořeného bodovými elektrickými náboji rozmístěných v
prostoru.

Program bude možné spustit z příkazové řádky skriptem \code{./run.sh} s
volitelným celočíselným parametrem určujícím scénář (viz níže) k vizualizaci.
Nebude-li parametr zadán, bude zvolen scénář 0. Po spuštění programu se zobrazí
okno o minimální počáteční velikosti 800x600px, ve kterém se zobrazí přehledně
všechny bodové náboje zvoleného scénáře. Vizuální reprezentace náboje (v
nejjednodušší podobě kružnice s popiskem) musí být volena tak, aby bylo možné
jednoznačně určit velikost náboje a jeho znaménko. V místě sondy bude zobrazen
šipkou vektor intenzity vektorového pole $\vec{E}$ a opatřen popiskem udávající
jeho velikost $\|\vec{E}\|$. Všechny elementy vizualizace (náboje, popisky,
šipky) musí být plně viditelné.

Intenzita vektorového pole v daném místě $\vec{x}$ se spočte
z Coulombova zákona jako:
\[ \vec{E}(\vec{x}) = \frac{1}{4 \pi \epsilon_0}
\sum_i q_i \frac{\vv{x_i - x}}{\|\vv{x_i - x}\|^3} \]

Vizualizace se bude pravidelně (s časem) aktualizovat. Jednotlivé elementy
vizualizace budou automaticky svou velikost a rozmístění přizpůsobovat velikosti
(a tvaru) okna tak, aby se prostor okna „maximálně“ využil, ale zároveň aby
nedošlo ke zkreslení souřadnic.

Scénáře:

\begin{itemize}
  \item Scénář 0 = pouze jeden bodový náboj +1C v počátku souřadnic (0, 0);
  \item Scénář 1 = jeden náboj +1C v (-1, 0), druhý +1C v (1, 0);
  \item Scénář 2 = jeden náboj -1C v (-1, 0), druhý +2C v (1, 0);
  \item Scénář 3 = čtyři náboje: +1C v (-1, -1), +2C v (1, -1), -3C v (1, 1)
    a -4C v (-1, 1);
\end{itemize}

Ve všech těchto scénářích se sonda pohybuje po kružnici se středem v počátku
souřadnic (0, 0), poloměrem 1 a úhlovou rychlostí \( \frac{\pi}{6}
\frac{rad}{s} \).

\section{Implementace řešení}

\subsection{Použité technologie}

Vizualizace je naprogramována v jazyce \CC{} 23 s využitím grafické knihovny
\href{https://www.raylib.com/}{raylib} verze 5.5, jejího portu pro jazyk \CC{}
\href{https://github.com/RobLoach/raylib-cpp}{raylib-cpp}, rozšíření pro
uživatelská rozhraní \href{https://github.com/raysan5/raygui}{raygui}, knihovny
pro zpracování formátu JSON
\href{https://github.com/nlohmann/json}{nlohmann/json} a knihovny pro fondy
vláken \href{https://github.com/progschj/ThreadPool}{ThreadPool}.

\subsection{Architektura}

Program využívá OOP architekturu, kde každý grafický a logický prvek je uzavřený
do své třídy a zodpovědný za veškeré výpočty a kreslení, které se ho týkají.
Funkce \code{main} má na starost pouze počáteční nastavení a následné předávání
událostí (nový snímek, změna velikosti okna, apod.) objektům. Některé objekty
jako součást obsahují další objekty a potom mají na starost jim události
předávat.

\subsection{Návrhová rozhodnutí}

Během implementace programu jsem musel udělat několik zásadních návrhových
rozhodnutí o věcech, která v zadání byla specifikovaná volněji. Asi
nejviditelnějším z nich je použití funkce sigmoid pro modulaci barev nábojů a
barevné mapy na pozadí. Důvodem byl velký rozsah hodnot, kterých můžou nabývat
(několik řádů), což sigmoid převede na rozumné hodnoty v rozsahu $-1.0$ až
$1.0$, díky čemuž pak barvy mají plynulejší přechod.

Druhé zásadní rozhodnutí bylo na základě inspirace ilustračním obrázkem použití
elektrického potenciálu namísto velikosti intenzity pro barevnou mapu a graf,
protože přenáší více informací a dle mého názoru vypadá zajímavěji. To mi (na
doporučení doc. Kohouta) potrvdilo i několik dotázaných respondentů.

Obecně byl kladen velký důraz na vizuální stranku programu, aby vše by pokud
možno plynulé a (opticky) spojité, např. že posun, zoom, velikost nábojů,
otevření grafu, apod. neskáčou a vše má dostatečnou snímkovou frekvenci - v
testovaní více než 60 Hz i na FullHD na slabším notebooku v režimu úspory
baterie (ve výchozím nastavení a scénáři 4).

\subsection{Objekty}

Každý objekt má především 2 metody \code{update} a \code{draw}. \code{update} je
zodpovědná za přepočítání vlastností objektu v závislosti na čase, popř. změně
vlastností jiných objektů (např. sonda mění polohu podle času a naměřenou
hodnotu podle síly a polohy nábojů).

Každý zde popsaný objekt se nachází ve svém vlastním souboru.

\subsubsection{\code{Charge}}

Reprezentuje v simulaci jeden náboj. 

Velikost náboje může být buď konstantní nebo proměnlivá v čase - reprezentováno
třídami \code{StaticStrength} a \code{VariableStrength}. 

Vykresluje se jako kruh, jehož plocha (nikoliv průměr) je přímo úměrná velikosti
náboje. Dále má bílé ohraničení a barvu výplně reprezentující velikost náboje -
červená 5 C a více, modrou 5 C a méně a lineární interpolaci přes tmavě šedou
pro hodnoty mezi tím. Nakonec je ve středu náboje zobrazen text s numerickou
hodnotou velikosti náboje.

Kolečkem myši je možné náboj zvětšit nebo zmenšit, tak že je měněn jeho
násobitel velikosti, tedy není to lineární změna.

\subsubsection{\code{Probe}}

Reprezentuje v simulaci elektrickou sondu.

Může být buď statická nebo otáčivě pohyblivá - reprezentováno třídami
\code{Static}, resp. \code{Rotating} ze jmenného prostředí \code{position}.

Vykresluje se jako šipka reprezentující směr. Délka šipky je pro přehlednost
pevně daná, protože pokusy s proměnnou délkou v závislosti na intenzitě pole se
neosvědčili. Intenzita se pohybuje v rozmezí několika řádů a velikost se tedy
pohybovala od sotva viditelné po více než 3 obrazovky dlouhé.

Zároveň může sonda volitelně zobrazovat text s numerickou velikostí intenzity ve
vědecké notaci.

\subsubsection{\code{Grid}}

Představuje mřížku na pozadí, včetně sond v jejích průsečících.

Má nezávisle nastavitelné rozestupy mezi čárami v obou směrech v pixelech - ve
výchozím nastavení to je 50x50. Je dobré nenastavovat příliš malé rozestupy,
protože pak je jednak obraz nepřehledný a druhak se velmi zvyšují nároky na
výpočetní výkon grafické karty (v testovaní mřížka 10x10 na WQHD monitoru
vytížila i high-end kartu).

\subsubsection{\code{HeatMap}}

Představuje vizualizaci intenzity vektorového pole na pozadí scény.

Přebírá funkci pro výpočet hodnoty pro jeden pixel (v rozmezí $-1$ až $1$), kterou
pak interně převádí na tří-hodnotový gradient barev, pomocí funkce
\code{lerpColor3}. Pro urychlení výpočtu počítá funkci v nižším rozlišení
(podvzorkovaní) - s hodnotou danou konstantou \code{BACKGROUND\_SUBSAMPLE}) a
při vykreslení obrázek roztáhne. Dále využívá pomocného module \code{parallel},
což umožnuje využívat pouze 2-násobné podvzorkování zajištující vysokou kvalitu
obrazu bez viditelných čtverečků i při zvětšení pomocí metody Nearest Neighbor.

Technická poznámka: Protože samotné renderování probíhá na grafické kartě, je
nutné předávat data nepřímo přes knihovní třídu \code{Texture} a její metodu
\code{UpdatePixels}, která zajistí jejich překopírování do grafické paměti.
Podobně je nutné tuto texturu mazat z znovu vytvářet při změně velikosti okna,
neboť má po alokaci fixní rozměry.

\subsubsection{\code{Plot}}
\label{sec:plot}

Třída pro vykreslování grafu intenzit pole zachycených uživatelskými sondami v
závislosti na čase. 

Každý snímek přebírá vektor sond, jejichž hodnoty si ukládá do vektoru front
(řad). K nim si ještě drží tzv. \code{offset}, který slouží pro správné
vykreslování grafů již odebraných sond. Více viz. popis rozšíření více sond.

Při vykreslování zvolí měřítko podle největší hodnoty ve všech řadách a všechny
řady podle toho škáluje. Tím dosáhne optimálního využití plochy na obrazovce.

\subsubsection{\code{FieldLine}}
\label{sec:fieldline}

Třída pro počítání a vykreslování siločar.

Siločáry počítá jednoduše iteračně tak, že začne v náboji (přesněji těsně vedle,
kvůli možnému dělení nulovou vzdáloností) a až 1000-krát se pohne ve směru
elektrického pole v daném bodě (jako by to byla kladná částice). Případně skončí
dříve pokud mezitím „narazí“ do jiného náboje nebo se dostane přiliš daleko mimo
obrazovku.

Siločáry vycházejí ve výchozím stavu z kladně nabitých částic, ale přepnou na
záporné, pokud je jich v simulaci více.

Počet siločar je řízen konstantou \code{LINES\_PER\_CHARGE} v souboru \code{defs}.

\subsection{Ostatní soubory}

\subsubsection{\code{defs}}

Definuje globální konstanty a proměnné pro běh a vzhled programu.

\subsubsection{\code{field}}

Jmenné prostředí obsahuje funkce \code{E} a \code{potential}, počítající vektor
intezity, resp. skalární potenciál, elektrického pole v zadaném bodě.

\subsubsection{\code{utils}}

Obsahuje pomocné funkce pro interpolaci barev \code{lerpColor} a \code{lerpColor3},
funkci \code{sigmoid} pro počítání stejnojmenné matematické funkce a funkci
\code{trim} pro odstranění počátečních a koncových bílých znaků z řetězce.

\subsubsection{\code{MathEval}}

Sada funkcí pro vyhodnocovaní libovolných matematických výrazů, využívaná pro
dynamicky nastavitelné vlastnosti.

Používá standardní kombinaci
\href{https://en.wikipedia.org/wiki/Shunting_yard_algorithm}{shunting yard}
algoritmu pro převedení výrazu do
\href{https://en.wikipedia.org/wiki/Reverse_Polish_notation}{reverse Polish
notation} a následného vyhodnocení zásobníkovým algoritmem. Při vyhodnocení
poskytuje konstanty $\pi$ a $e$, proměnnou času $t$ a funkce sin, cos, tan,
asin, acos, atan, log, exp, sqrt, abs, sign, floor a ceil.

Původně zamýšleno pro implementaci editoru dynamických scénářů, nakonec využito
pouze pro načítání ze souborů, protože už jsem nasbíral dostatek bodů jinými
rozšířeními.

\subsubsection{\code{parallel}}

Sada pomocných funkcí pro spouštění kódu s využitím všech jader procesoru.

Obaluje knihovnu \href{https://github.com/progschj/ThreadPool}{ThreadPool} do
rozhraní kompatibilního s \CC{} 23 modulem \code{ranges}, což umožnuje velmi
snadnou a pamětově bezpečnou paralelizaci libovolných cyklů prostřednictvím
funkce \code{for\_each}.

Poznámka: \CC{} teoreticky podporuje paralelizaci ve standardní funkci
\code{std::for\_each}, ale to vyžaduje externí knihovny a velmi nové verze
překladače.

\subsection{Rozšíření}

\subsubsection{Načtení scénáře ze souboru}

Lze načítat scénáře ze složky \code{scenarios} ve formátu JSON. V základu
program obsahuje 5 ukázkových scénářů podle zadání, které lze libovolně upravit
nebo doplnit vlastní. V příkazové řádce je pak potřeba specifikovat název
souboru ve složce \code{scenarios} \textit{bez cesty a přípony} podobně jako
vestavěné scénáře.

JSON soubor obsahuje 2 seznamy: \code{charges} a \code{probes}. \code{charges}
obsahuje objekty s atributem \code{strength}, který musí být buď číslo nebo
řetězec vyhodnotitelný jako funkce $f(t)$, a attributem position s atributy
\code{x} a \code{y} popisující pozici náboje. \code{probes} obsahuje objekty s
atributy \code{position} (opět s atributy \code{x} a \code{y}) a volitelně
atributem \code{rotation} s atributy \code{radius} a \code{velocity} popisující
poloměr otáčení a úhlovou rychlost v jednotkách $\frac{rad}{s}$ respektive.

\subsubsection{Zrychlení/zpomalení simulace}

Tlačítka v pravém dolním rozhu umožnují simulaci až 4-krát zrychlit či zpomalit.
Řešeno je to pomocí přenásobení délky snímků před přičtením k promenné času.

\subsubsection{Zvětšení a posun}

Simulaci lze zvětšit pomocí kolečka myši spolu s podržením tlačítka
\keystroke{Shift}. Realizován je pomocí nastavení parametru \code{zoom} kamery,
což odpovídá použití transformace škálování na celou scénu. 

Podobně posun lze provést držením pravého tlačítka myši. Realizován je pomocí
nastavení parametru \code{target} kamery, což odpovídá použití inverzní
transformace posunutí na celou scénu. Posunutí je pro přehlednost omezeno, aby
nebylo možné se „ztratit“ mimo viditelnou scénu.

\subsubsection{Siločáry}

Detaily jsou uvedeny v popisu zodpovědné třídy
\hyperref[sec:fieldline]{\code{FieldLine}}.

\subsubsection{Více sond}

Další sondy lze přidat levým tlačítkem myši, stejně jako první sondu. Další
sondy mají paletu 5 možných barev, po jejichž vyčerpání se generují další barvy
náhodně.

Další detaily jsou uvedeny v popisu zodpovědné třídy
\hyperref[sec:plot]{\code{Plot}}.

\section{Ovládání programu}

\subsection{Sestavení}

Program je kompatibilní s GNU/Linux systémy s podporou grafického protokolu X11.

Sestavení se provádí pomocí přiloženého \code{Makefile} příkazem \code{make
setup build}.

Protože ale mnoho systémů nemá nainstalované hlavičkové soubory pro knihovnu X11
(potřebné pro vývoj většiny grafických aplikací), má zastaralé verze překladačů,
apod. zvolil jsem použití systému Docker. Sestavovací skript \code{./build.sh}
vytvoří kontejner z přiloženého souboru \code{Dockerfile} a spustí ho. Ten 
spustí přiložený \code{Makefile}, který do složky \code{vendor} stáhne knihovny
a vše sestaví do složky \code{build}. Skript pak ještě přesune výsledný
spustitelný soubor do souboru \code{bin/elektroviz.new}.

\subsection{Spuštění}

Program přebírá z příkazové řádky parametr \code{scenario} a volitelně
\code{spacing}. \code{scenario} Určuje který scénář má být spuštěn po zapnutí
aplikace, při absenci parametru se použije scénář 0. \code{spacing} určuje
rozestupy mřížky na pozadí ve formátu \code{-g<width>x<height>}.

\textbf{Důležité}: Program je nutné spustit ve stejné složce, v jaké se nechází
složka \code{scenarios}, jinak nebude schopný načíst žadný z výchozích scénářů.
Originálně je ve složce \code{src}, ale \code{./build.sh} jí pro snadnější
manipulaci nakopíruje i do kořenové složky.

\subsection{Uživatelské vstupy}  

\begin{itemize}
  \item levé tlačítko myši: přidání nové sondy, grafická tlačítka
  \item levé tlačítko myši + \keystroke{Shift}: odebrání sondy
  \item pravé tlačítko myši (držení): posun scény
  \item kolečko myši: zvětšení/zmenšení scény
  \item kolečko myši (držení): posun nábojů
  \item kolečko myši + \keystroke{Shift}: změna velikosti nábojů
  \item \keystroke{Esc}: ukončení aplikace
\end{itemize}

\section{Limitace}

Program pouze minimálně ošetřuje nesprávné uživatelské vstupy a reaguje na ně
vyjímkami, které můžou vést až k pádu aplikace. 

Také není oficiálně podporována možnost, kdy náboj s velikostí danou funkcí, mění
v průběhu času znaménko. Fyzikálně vše funguje správně, ale je možný výskyt
grafických artefaktů a problikávaní.

\section{Poznámky k dokumentaci}

Šablona pro dokumentaci je převzatá z předmětu KKY/HKUI a doplněná o grafické
prvky z fakultní šablony FASThesis.

Strávený čas je měřený prostřednictvím IDE, tedy odpovídá poměrně přesně
realitě. Zahrnuje ale i učení se knihovny, jazyka \CC{}, experimentaci se
shadery a výpočty na GPU a další věci mimo rámec předmětu. Čistý čas stravený na
projektu je tedy o několik desítek hodin menší.

\end{document}
